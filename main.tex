\documentclass{article}

\title{S\'eparation \'Eglise-\^Etat d\^u \`a la Science}
\author{Thomasss??, Ogier Bouvier, Jean-Thomas Furrer, Jo\"el Kaufmann, Val\'erian Rousset}

\begin{document}

\maketitle

\section{Introduction}

\section{Rappel historique}

La diff\'erence entre \'eglise et science n'est qu'un questionnement recent, qui ne
s'est pr\'esent\'e qu'aux environs du 16\textsuperscript{\`eme} si\`ecle. Durant l'antiquit\'e, les pr\'edicateurs
\'etait souvent aussi des savants, bien que ce terme ne soit pas encore usit\'e
mais \'etaient plutôt nomm\'es sage ou savant. Ainsi, Pythagore fut pr\^etre, et m\^eme
s'il trouvait parfois des incoh\'erences avec le dogme courant, il ne s'\'erigeait
pourtant pas en vindicateur contestataire. Il y avait alors plus de tensions
interreligieuses qu'intrareligieuses.

Pour ne pas rester centrer sur l'Europe, il est bon de noter qu'il y eu un
parall\`ele aussi en Orient, où la Science, dont les parties les plus nobles
\'etaient l'astrologie et l'alchimie, perdura jusqu'au 12\textsuperscript{\`eme} si\`ecle.
Certain des textes grecs furent retrouv\'e par l'Occident \`a la fin de cette
p\'eriode et traduit en latin, et amen\`erent un v\'eritable fleuve au moulin des
penseurs raisonn\'es, qui insistaient fortement sur l'importance de la r\'eflexion
et la logique, comme moyen de s'approcher de Dieu en attendant une r\'ev\'elation
promise.

\section{Conclusion}

\begin{thebibliography}{9}

\bibitem{article}
	Jean-Louis Schlegel,
	\emph{Religion et s\'ecularisation: Science et religion},
	Les religions dans la soci\'et\'e,
	Cahiers français n°340

\end{thebibliography}


\end{document}
