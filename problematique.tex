\documentclass{article}

\title{Séparation Église-Êtat dû à la Science \\
	\large{Problématique}}
\author{Thomas Batschelet, Ogier Bouvier, Jean-Thomas Furrer, \\ Joël Kaufmann, Valérian Rousset}

\begin{document}

\maketitle

\section{Aperçu}

La problématique que nous voulons exposer lors de notre mémoire, est la
suivante: Dans quel mesure les avancées de la Science ont influencé la séparation
entre l'Église et l'État. Nous allons l'étudier tout d'abord sous l'angle
historique, voir quelle est la relation entre la Science et les Religions,
était-ce déjà séparé simplement, ou les principaux scientifiques de l'époque
étaient aussi prédicateurs? Nous allons après passer en profondeur sur les
débuts de la réforme et étudier les dissensions créées alors et les débuts de
la différenciation entre Église et l'État.

Puis, le moment central arrive, la substance de ce mémoire se trouve développé
plus bas, mais nous en donnons un bref aperçu ici. L'idée principale est de voir
comment et pourquoi y a-t'il eu un changement, une transition entre l'Église
encore mêlée à l'État vers une séparation lente mais clair, du entre autre à la
réforme, le siècle des lumières et les différentes révolutions Européen.

\section{Développement de la problématique}

"Dans quel mesure les avancées de la Science ont influencé la séparation
entre l'Église et l'État"

\subsection{Science}

Comment a été défini la Science? Depuis quand existe-elle? A-t'elle un passé
religieux et dans ce cas, comment étaient relié les deux? Pourquoi et comment
y'a-t'il eu possibilité de séparation avec l'Église? Y'a-t'il eu un mouvement
religieux amenant à cette ouverture? Était-ce nécessaire pour faire revivre un
sentiment croyant plus fort?

\subsection{Église}

De quand date la constitution de l'Église? Il y avait-il précédemment une idée
scientifique ou les prêtres étaient-ils aussi des savants? Quelles ont été les
dissensions entre les deux et comment cela a-t'il permit une évolution commune?

\subsection{État}

Quel est l'événement fondateur de cet État? Comment les différents courants
politique et théorique ont amenés des modifications dans cette vision de l'État,
et la relation de cet État avec la religion? Comment la Science a été modifié,
transformé par cet État? Et comment cette Science a amené des changements la
relation État et Église? Les États se sont-ils servit de cette Science pour
réfuter la religion? Quels sont les courants politiques qui se sont servi de la
Science pour s'éloigner de l'Église?

\subsection{Evolution naturelle}

Cette séparation entre Église et état est-elle une évolution naturelle
de l'être humain comme voudrait le sous-entendre Auguste Comte avec sa
``loi des 3 états''? Pour rappel, celle ci postule en effet que l'être humain
passerait par 3 stades successifs de pensée chacun prenant plus de
distance avec la pensée théologique. Cette loi des 3 états est-elle
fondée dans la réalité et représentative de l'état actuel des
différentes cultures et sociétés humaines? Ou est-elle simplement une
construction abstraite déconnecté de la réalité?

\section{Résumé}

Nous allons donc faire premièrement une analyse historique en profondeur des
origines des différèrent éléments en jeu, depuis leur création jusqu'à
l'utilisation dans notre mémoire. Nous allons aussi décrire la rencontre et les
querelles qui sont advenue cet impact.

\end{document}
