\documentclass{article}

\title{Séparation Église-État due à la Science \\
	\large{Problématique}}
\author{Thomas Batschelet, Ogier Bouvier, Jean-Thomas Furrer, \\ Joël Kaufmann, Valérian Rousset}

\begin{document}

\maketitle

\section{Aperçu}

La problématique que nous voulons exposer lors de notre mémoire, est la
suivante: Dans quelle mesure les avancées de la Science ont influencé la séparation
entre Église et État. Nous allons l'étudier tout d'abord sous l'angle
historique, c'est à dire étudier la relation entre la Science et les Religions.
Pouvait-on déjà les différentier à l'époque, sachant que les principaux scientifiques
étaient aussi prédicateurs? Depuis quand fait-on la distinction entre les deux?
Nous allons ensuite passer en profondeur sur les débuts de la réforme et étudier les dissensions
alors créées et les débuts de la différenciation entre Église et État.

Puis, le moment central arrive, la substance de ce mémoire se trouve développée
plus bas, mais nous en donnons un bref aperçu ici. L'idée principale est de voir
comment et pourquoi y a-t-il eu un changement, une transition entre l'Église
encore mêlée à l'État vers une séparation lente mais claire due, entre autre, à la
réforme, le siècle des lumières et les différentes révolutions Européen.

\section{Développement de la problématique}

"Dans quelle mesure les avancées de la Science ont influencé la séparation
entre l'Église et l'État"

\subsection{Science}

Comment a été définie la Science? Depuis quand existe-t-elle? A-t-elle un passé
religieux et, dans ce cas, comment étaient reliés les deux? Pourquoi et comment
est apparue la possibilité de séparation avec l'Église? Y'a-t-il eu un mouvement
religieux amenant à cette ouverture? Était-ce nécessaire pour faire revivre un
sentiment croyant plus fort?

\subsection{Église}

De quand date la constitution de l'Église? Y avait-il précédemment une idée
scientifique ou les prêtres étaient-ils simplement aussi des savants? Quelles ont été les
dissensions entre les deux et comment cela a-t-il permis une évolution commune?

\subsection{État}

Quel est l'événement fondateur de cet État? Comment les différents courants
politique et théorique ont amenés des modifications dans cette vision de l'État,
et la relation de l'État avec la religion? Comment la Science a-t-elle été modifiée,
transformée par l'État? Et comment cette Science a amené des changements à la
relation État et Église? Les États se sont-ils servis de la Science pour
réfuter la religion? Quels sont les courants politiques qui se sont servis de la
Science pour s'éloigner de l'Église?

\subsection{Evolution naturelle}

Cette séparation entre Église et État est-elle une évolution naturelle
de l'être humain comme voudrait le sous-entendre Auguste Comte avec sa
``loi des 3 états''? Pour rappel, celle-ci postule en effet que l'être humain
passerait par 3 stades successifs de pensée chacun prenant plus de
distance avec la pensée théologique. Cette loi des 3 états est-elle
fondée dans la réalité et représentative de l'état actuel des
différentes cultures et sociétés humaines? Ou est-elle simplement une
construction abstraite déconnectée de la réalité?

\section{Résumé}

Nous allons donc faire premièrement une analyse historique en profondeur des
origines des différents éléments en jeu, depuis leur création jusqu'à
leur implication. Nous allons aussi décrire la rencontre et les
querelles qui se sont dégagées de cette confrontation entre sciences et religions.

\end{document}
