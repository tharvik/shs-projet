\documentclass{article}

\title{S\'eparation \'Eglise-\^Etat d\^u \`a la Science \\
	\large{Probl\'ematique}}
\author{Thomas Batschelet, Ogier Bouvier, Jean-Thomas Furrer, \\ Jo\"el Kaufmann, Val\'erian Rousset}

\begin{document}

\maketitle

\section{Aper\c{c}u}

La probl\'ematique que nous voulons exposer lors de notre m\'emoire, est la
suivante: Dans quel mesure les avanc\'ees de la Science ont influenc\'e la s\'eparation
entre l'\'Eglise et l'\'Etat. Nous allons l'\'etudier tout d'abord sous l'angle
historique, voir quelle est la relation entre la Science et les Religions,
\'etait-ce d\'ej\`a s\'epar\'e simplement, ou les principaux scientifiques de l'\'epoque
\'etaient aussi pr\'edicateurs? Nous allons apr\`es passer en profondeur sur les
d\'ebuts de la r\'eforme et \'etudier les dissentions cr\'e\'ees alors et les d\'ebuts de
la diff\'erenciation entre Église et l'État.

TODO

\section{D\'eveloppement de la probl\'ematique}

"Dans quel mesure les avanc\'ees de la Science ont influenc\'e la s\'eparation
entre l'\'Eglise et l'\'Etat"

\subsection{Science}

Comment a \'et\'e d\'efini la Science? Depuis quand existe-elle? A-t'elle un pass\'e
religieux et dans ce cas, comment \'etaient reli\'e les deux? Pourquoi et comment
y'a-t'il eu possibilit\'e de s\'eparation avec l'Église? Y'a-t'il eu un mouvement
religieux amenant \`a cette ouverture? Était-ce n\'ecessaire pour faire revivre un
sentiment croyant plus fort?

\subsection{Église}

De quand date la constitution de l'Église? Il y avait-il pr\'ec\'edemment une id\'ee
scientifique ou les pr\^etres \'etaient-ils aussi des savants? Quelles ont \'et\'e les
dissensions entre les deux et comment cela a-t'il permit une \'evolution commune?

\subsection{État}

Quel est l'\'ev\'enement fondateur de cet État? Comment les diff\'erents courants
politique et th\'eorique ont amen\'es des modifications dans cette vision de l'État,
et la relation de cet État avec la religion? Comment la Science a \'et\'e modifi\'e,
transform\'e par cet État? Et comment cette Science a amen\'e des changements la
relation État et Église? Les États se sont-ils servit de cette Science pour
r\'efuter la religion? Quels sont les courants politiques qui se sont servi de la
Science pour s'\'eloigner de l'Église?

\section{R\'esum\'e}

Nous allons donc faire premi\`erement une analyse historique en profondeur des
origines des diff\'er\`erent \'el\'ements en jeu, depuis leur cr\'eation jusqu'\`a
l'utilisation dans notre m\'emoire. Nous allons aussi d\'ecrire la rencontre et les
querelles qui sont advenue cet impact.

\end{document}
