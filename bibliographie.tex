\documentclass{article}

\begin{document}
\begin{thebibliography}{9}

		\bibitem{article}
		\textbf{- La science et la religion, quels liens? Concordisme ou désaccord?}\\
		Jean-Louis Schlegel,
		\emph{Religion et s\'ecularisation: Science et religion},
		Les religions dans la soci\'et\'e,
		Cahiers français n°340
		\\\\
		Jacqueline Lagrée, \emph{La modernité contre la religion ? : pour une nouvelle approche de la laïcité}, Presses universitaires de Rennes, Rennes, 2010
		\\\\
		\textbf{- Aspects historiques de la laïcité française. Dans quel context? Pourquoi?}\\
		Jean-Marie Mayeur,
		\emph{La Séparation de l'Église et de l'État}, Paris, Colin,
		Collection archives Julliard, 1998, réédition
		rééd. avec compléments, Éditions de l'Atelier, 2005.
		\\\\
		Guillaume Carnino, \emph{Science et religion : la séparation constitutive} in \emph{L'Invention de la science : La nouvelle religion de l'âge industriel}, Seuil, Paris, 2005, pp. 61-86
		\\\\
		André Damien, Yves Bruley, Dominique de Villepin et Jean-Michel Gaillard, \emph{1905, la séparation de l'Église et de l'État : Les textes fondateurs}, Librairie Académique Perrin, Paris, 2004.
		\\\\
		Jean Baubérot, \emph{Histoire de la laïcité en France}, Que sais-je n° 3571, Presses Universitaires de France, 2010.
		\\\\
		Jean Tulard, Yves Bruley, \emph{Histoire de la laïcité à la française : loi de 1905: Le livre du centenaire officiel}, Académie des sciences morales et poltiques, Paris, 2005
		\\\\
		\textbf{- Sciences et religions : a-t-on besoin de la religion pour rester éthique?}\\
		Jacques Arnould,
		\emph{Sciences et laïcité},
		Revue d'éthique et de théologie morale, 2008, Issue 1, p.106
		\\\\
		\textbf{- L'évolution de la laïcité et ses critiques}\\
		Pierre Hayat,  \emph{La laïcité et les pouvoirs pour une critique de la raison laïque}, Kime, 1998.
		\\\\
		\textbf{- Esprit positif : La séparation science-religion coule-t-elle de source? Vers la fin des religions?}\\
		Auguste Comte, \emph{Discours sur l'esprit positif (1844)}, Vrin, Paris, 1995\\
		https://fr.wikipedia.org/wiki/Positivisme\\
		https://fr.wikipedia.org/wiki/Loi\_des\_trois\_\%C3\%A9tats\\
		http://classiques.uqac.ca/classiques/Comte\_auguste/discours\_esprit\_positif/Discours\_esprit\_positif.pdf

\end{thebibliography}


\end{document}
