\documentclass{article}
\usepackage[frenchb]{babel}
\begin{document}
	
	\begin{thebibliography}
		
		\bibitem{article}
		\textbf{- La science et la religion, quels liens? Concordisme ou d�saccord?}\\
		Jean-Louis Schlegel,
		\emph{Religion et s\'ecularisation: Science et religion},
		Les religions dans la soci\'et\'e,
		Cahiers fran�ais n�340
		\\\\
		Jacqueline Lagr�e, \emph{La modernit� contre la religion ? : pour une nouvelle approche de la la�cit�}, Presses universitaires de Rennes, Rennes, 2010
		\\\\
		\textbf{- Aspects historiques de la la�cit� fran�aise. Dans quel context? Pourquoi?}\\
		Jean-Marie Mayeur,
		\emph{La S�paration de l'�glise et de l'�tat}, Paris, Colin, Collection \og archives Julliard \fg, 1998, r��d. avec compl�ments, �ditions de l'Atelier, 2005.
		\\\\
		Guillaume Carnino, \emph{Science et religion : la s�paration constitutive} in \emph{L'Invention de la science : La nouvelle religion de l'�ge industriel}, Seuil, Paris, 2005, pp. 61-86
		\\\\
		Andr� Damien, Yves Bruley, Dominique de Villepin et Jean-Michel Gaillard, \emph{1905, la s�paration de l'�glise et de l'�tat : Les textes fondateurs}, Librairie Acad�mique Perrin, Paris, 2004.
		\\\\
		Jean Baub�rot, \emph{Histoire de la la�cit� en France}, Que sais-je n� 3571, Presses Universitaires de France, 2010.
		\\\\
		Jean Tulard, Yves Bruley, \emph{Histoire de la la�cit� � la fran�aise : loi de 1905: Le livre du centenaire officiel}, Acad�mie des sciences morales et poltiques, Paris, 2005
		\\\\
		\textbf{- Sciences et religions : a-t-on besoin de la religion pour rester �thique?}\\
		Jacques Arnould,
		\emph{Sciences et la�cit�},
		Revue d'�thique et de th�ologie morale, 2008, Issue 1, p.106
		\\\\
		\textbf{- L'�volution de la la�cit� et ses critiques}\\
		Pierre Hayat,  \emph{La la�cit� et les pouvoirs pour une critique de la raison la�que}, Kime, 1998.
		\\\\
		\textbf{- Esprit positif : La s�paration science-religion coule-t-elle de source? Vers la fin des religions?}\\
		Auguste Comte, \emph{Discours sur l'esprit positif (1844)}, Vrin, Paris, 1995\\
		https://fr.wikipedia.org/wiki/Positivisme\\
		https://fr.wikipedia.org/wiki/Loi\_des\_trois\_\%C3\%A9tats\\
		http://classiques.uqac.ca/classiques/Comte\_auguste/discours\_esprit\_positif/Discours\_esprit\_positif.pdf
	\end{thebibliography}	
\end{document}
