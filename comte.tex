\documentclass{article}

\title{Séparation Église-Êtat dû à la Science \\
  \large{Problématique}}
\author{Thomas Batschelet, Ogier Bouvier, Jean-Thomas Furrer, \\ Joël Kaufmann, Valérian Rousset}

\begin{document}

\maketitle

Pour certains, comme Auguste Comte, cette séparation est inéluctable
et est seulement le résultat naturel de l'évolution humaine. En effet
selon Comte l'homme passerait par 3 états successifs, l'état
théologique d'abord, suivi ensuite l'état métaphysique puis
l'aboutissement final, l'état scientifique. Pour Comte, la science
joue donc un rôle majeur (peut-être même est seule responsable) de la
séparation entre église et état. Cette séparation doit pour Comte
résulter en une politique fondée sur une organisation rationnelle de
la société ce qui contredirait l'organisation traditionnelle
occidentale telle qu'elle fût jusqu'en aux alentours du
18\textsuperscript{ième} siècle en Europe. On peut en effet voir un
exemple assez flagrant de cette loi dans l'histoire française.

\end{document}
